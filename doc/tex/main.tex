%%%%%%%%%%%%%%%%%%%%%%%%%%%%%%%%%%%%%%%%%%%%%%%%%%%%%%%%%%%%%%%%%%%%%%%%
%     LaTeX source code to approximate a NIST Technical report
%	  Instructions for authors: tinyurl.com/techpubsnist 
%	DOI watermark will be added on final PDF
% 	Developed by K. Miller, kmm5@nist.gov 
%	Last updated: 26-March-2019
%%%%%%%%%%%%%%%%%%%%%%%%%%%%%%%%%%%%%%%%%%%%%%%%%%%%%%%%%%%%%%%%%%%
\documentclass[12pt]{article}
\usepackage{amsmath}
\usepackage{amsfonts}   % if you want the fonts
\usepackage{amssymb}    % if you want extra symbols
\usepackage{graphicx}   % need for figures
\usepackage{xcolor}
\usepackage{bm}
\usepackage{secdot}		
\usepackage{mathptmx}
\usepackage{float}
\usepackage[utf8]{inputenc}
\usepackage{textcomp}
\usepackage[hang,flushmargin,bottom]{footmisc} % footnote format

\usepackage{titlesec}
\titleformat{\section}{\normalsize\bfseries}{\thesection.}{1em}{}	% required for heading numbering style
\titleformat*{\subsection}{\normalsize\bfseries}

\usepackage{tocloft}	% change typeset, titles, and format list of appendices/figures/tables
\renewcommand{\cftdot}{}	
\renewcommand{\contentsname}{Table of Contents}
\renewcommand{\cftpartleader}{\cftdotfill{\cftdotsep}} % for parts
\renewcommand{\cftsecleader}{\cftdotfill{\cftdotsep}}
\renewcommand\cftbeforesecskip{\setlength{4pt}{}}
\addtolength{\cftfignumwidth}{1em}
\renewcommand{\cftfigpresnum}{\figurename\ }
\addtolength{\cfttabnumwidth}{1em}
\renewcommand{\cfttabpresnum}{\tablename\ }
\setlength{\cfttabindent}{0in}    %% adjust as you like
\setlength{\cftfigindent}{0in} 

\usepackage{enumitem}         % to control spacing between bullets/numbered lists

\usepackage[numbers,sort&compress]{natbib} % format bibliography 
\renewcommand{\bibsection}{}
\setlength{\bibsep}{0.0pt}

\usepackage[hidelinks]{hyperref}
\hypersetup{
	colorlinks = true,
urlcolor ={blue},
citecolor = {.},
linkcolor = {.},
anchorcolor = {.},
filecolor = {.},
menucolor = {.},
runcolor = {.}
pdftitle={},%%put title here to auto-fill properties of the PDF
pdfsubject={},%%put abstract here
pdfauthor={}, %%put author list here
pdfkeywords={} %%put keywords here
}
\urlstyle{same}

\usepackage{epstopdf} % converting EPS figure files to PDF

\usepackage{fancyhdr, lastpage}	% formatting document, calculating number of pages, formatting headers
\setlength{\topmargin}{-0.5in}
\setlength{\headheight}{39pt}
\setlength{\oddsidemargin}{0.25in}
\setlength{\evensidemargin}{0.25in}
\setlength{\textwidth}{6.0in}
\setlength{\textheight}{8.5in}

\usepackage{caption} % required for Figure labels
\captionsetup{font=small,labelfont=bf,figurename=Fig.,labelsep=period,justification=raggedright} 

%%%%%%%%%%% !!!!!! REQUIRED - FILL OUT METADATA HERE !!!!!!!! %%%%%%%%%%%%%%
%  	Report Number - fill in Report Number sent to you (see info below)
%   DOI Statement - fill in DOI sent to you 
%   Month Year - fill in Month and Year of Publication
%%%%%%%%%%%%%%%%%%%%%%%%%%%%%%%%%%%%%%%%%%%%%%%%%%%%%%%%%%%%%%%%%%%%%%%%%%%%%%%%%%%%%%
\newcommand{\pubnumber}{0.0.1}
\newcommand{\DOI}{https://doi.org/10.6028/NIST.TN.XXXX}
\newcommand{\monthyear}{Month Year}
%%%%%%%%%%%%%%%%%%%%%%%%%%%%%%%%%%%%%%%%%%%%%%%%%%%%%%%%%%%%%%%%%%%%
%   	BEGIN DOCUMENT 
%%%%%%%%%%%%%%%%%%%%%%%%%%%%%%%%%%%%%%%%%%%%%%%%%%%%%%%%%%%%%%%%%%%%
\begin{document}
	\urlstyle{rm} % Format style of \url   
	
%%%%%%%%%%%%%%%%%%%%%%%%%%%%%%%%%%%%%%%%%%%%%%%%%%%%%%%%%%%%%%%%%%%%
%   Cover Page is REQUIRED and must contain the information 
%	displayed here, at a minimum. Additional artwork may be included 
%	(e.g., official project/conference logo, etc.).
%	Pub Number automated based on metadata
%%%%%%%%%%%%%%%%%%%%%%%%%%%%%%%%%%%%%%%%%%%%%%%%%%%%%%%%%%%%%%%%%%%%
	\begin{titlepage}
		\begin{flushright}
%%%%%%%%%%%%%%%%%%%%%%%%%%%%%%%%%%%%%%%%%%%%%%%%%%%%%%%%%%%%%%%%%%%%
% 	Automated based on metadata - delete if not applicable
%%%%%%%%%%%%%%%%%%%%%%%%%%%%%%%%%%%%%%%%%%%%%%%%%%%%%%%%%%%%%%%%%%%%
\LARGE{\textbf{DSLAM Technical Note \pubnumber}}\\
\vfill
%%%%%%%%%%%%%%%%%%%%%%%%%%%%%%%%%%%%%%%%%%%%%%%%%%%%%%%%%%%%%%%%%%%%
%	Title 
%%%%%%%%%%%%%%%%%%%%%%%%%%%%%%%%%%%%%%%%%%%%%%%%%%%%%%%%%%%%%%%%%%%%
\LARGE{\textbf{P2PDB: A Secure, Scalable, Decentralized Data Sharing and Management Platform}}\\
\vfill
%%%%%%%%%%%%%%%%%%%%%%%%%%%%%%%%%%%%%%%%%%%%%%%%%%%%%%%%%%%%%%%%%%%%
%	Authors - add complete list of authors, affiliations will be 
%   added on title page
%%%%%%%%%%%%%%%%%%%%%%%%%%%%%%%%%%%%%%%%%%%%%%%%%%%%%%%%%%%%%%%%%%%%
\large Gang Liao\\
\large Daniel J. Abadi\\
% \large Etc.\\
\textit{University of Maryland, College Park}\\
\vfill
%%%%%%%%%%%%%%%%%%%%%%%%%%%%%%%%%%%%%%%%%%%%%%%%%%%%%%%%%%%%%%%%%%%%
%	The DOI is automated based on metadata.	
%%%%%%%%%%%%%%%%%%%%%%%%%%%%%%%%%%%%%%%%%%%%%%%%%%%%%%%%%%%%%%%%%%%%
\normalsize The use of this publication is permitted only internally within DSLAM:\\
\url{http://dslam.cs.umd.edu}\\
\vfill
%%%%%%%%%%%%%%%%%%%%%%%%%%%%%%%%%%%%%%%%%%%%%%%%%%%%%%%%%%%%%%%%%%%%
%	NIST LOGO - keep as-is
%%%%%%%%%%%%%%%%%%%%%%%%%%%%%%%%%%%%%%%%%%%%%%%%%%%%%%%%%%%%%%%%%%%%

\includegraphics[width=0.2\linewidth]{img.jpg}\\ 


\end{flushright}
\end{titlepage}
% \begin{titlepage}
% %%%%%%%%%%%%%%%%%%%%%%%%%%%%%%%%%%%%%%%%%%%%%%%%%%%%%%%%%%%%%%%%%%%%
% %	Title Page is REQUIRED
% %%%%%%%%%%%%%%%%%%%%%%%%%%%%%%%%%%%%%%%%%%%%%%%%%%%%%%%%%%%%%%%%%%%%
% \begin{flushright}
% %%%%%%%%%%%%%%%%%%%%%%%%%%%%%%%%%%%%%%%%%%%%%%%%%%%%%%%%%%%%%%%%%%%%
% %   Publication Series & Number - automated
% %%%%%%%%%%%%%%%%%%%%%%%%%%%%%%%%%%%%%%%%%%%%%%%%%%%%%%%%%%%%%%%%%%%%
% \LARGE{\textbf{DSLAM Technical Note \pubnumber}}\\
% \vfill 
% %%%%%%%%%%%%%%%%%%%%%%%%%%%%%%%%%%%%%%%%%%%%%%%%%%%%%%%%%%%%%%%%%%%%
% %	Title 
% %%%%%%%%%%%%%%%%%%%%%%%%%%%%%%%%%%%%%%%%%%%%%%%%%%%%%%%%%%%%%%%%%%%%
% \Huge{\textbf{P2PDB: A Secure, Scalable, Decentralized Data Sharing and Management Platform}}\\
% \vfill
%%%%%%%%%%%%%%%%%%%%%%%%%%%%%%%%%%%%%%%%%%%%%%%%%%%%%%%%%%%%%%%%%%%%
%	Author Order and Grouping. Always identify the primary author/creator first (s/he does not have to be a NIST author). For publications with multiple authors, group authors by their organizational affiliation. The organizational groupings and the names within each grouping should generally be ordered by decreasing level of contribution.
%	For non-NIST authors, list their city and state below their organization name.
%	For NIST authors, include the Division and Laboratory names (but do not include their city and state).
%%%%%%%%%%%%%%%%%%%%%%%%%%%%%%%%%%%%%%%%%%%%%%%%%%%%%%%%%%%%%%%%%%%%
% \normalsize Gang Liao\\
% Daniel J. Abadi\\
% \textit{DSLAM - Data Systems Lab at Maryland}\\
% \textit{University of Maryland, College Park}\\
% % \vspace{12pt}
% % Third Author\\
% % Fourth Author\\
% % \textit{Office of XXXX}\\
% % \textit{Second Operating Unit}\\
% \vfill
%%%%%%%%%%%%%%%%%%%%%%%%%%%%%%%%%%%%%%%%%%%%%%%%%%%%%%%%%%%%%%%%%%%%
%   DOI Statement - automated
%%%%%%%%%%%%%%%%%%%%%%%%%%%%%%%%%%%%%%%%%%%%%%%%%%%%%%%%%%%%%%%%%%%%
% \normalsize This publication is available free of charge from:\\
% \DOI\\
% \vfill
% %%%%%%%%%%%%%%%%%%%%%%%%%%%%%%%%%%%%%%%%%%%%%%%%%%%%%%%%%%%%%%%%%%%%
% %   Date - Month and Year - automated
% %%%%%%%%%%%%%%%%%%%%%%%%%%%%%%%%%%%%%%%%%%%%%%%%%%%%%%%%%%%%%%%%%%%%
% \normalsize \monthyear
% \vfill
% %%%%%%%%%%%%%%%%%%%%%%%%%%%%%%%%%%%%%%%%%%%%%%%%%%%%%%%%%%%%%%%%%%%%
% %  Department of Commerce LOGO - leave as-is
% %%%%%%%%%%%%%%%%%%%%%%%%%%%%%%%%%%%%%%%%%%%%%%%%%%%%%%%%%%%%%%%%%%%%	

% \includegraphics[width=0.18\linewidth]{DoC-logo.eps}\\ 
% \vfill
% %%%%%%%%%%%%%%%%%%%%%%%%%%%%%%%%%%%%%%%%%%%%%%%%%%%%%%%%%%%%%%%%%%%%
% %  Department of Commerce & NIST Leadership 
% %	will be updated as changes occur
% %%%%%%%%%%%%%%%%%%%%%%%%%%%%%%%%%%%%%%%%%%%%%%%%%%%%%%%%%%%%%%%%%%%%
% \footnotesize U.S. Department of Commerce\\ 
% \textit{Wilbur L. Ross, Jr., Secretary}\\
% \vspace{10pt}
% National Institute of Standards and Technology\\ 
% \textit{Walter Copan, NIST Director and Undersecretary of Commerce for Standards and Technology}  
% \end{flushright}
% \end{titlepage}

% \begin{titlepage}
% %%%%%%%%%%%%%%%%%%%%%%%%%%%%%%%%%%%%%%%%%%%%%%%%%%%%%%%%%%%%%%%%%%%%
% %   Disclaimer/CODEN page - required
% %%%%%%%%%%%%%%%%%%%%%%%%%%%%%%%%%%%%%%%%%%%%%%%%%%%%%%%%%%%%%%%%%%%%
% \begin{flushright}
% \footnotesize  Certain commercial entities, equipment, or materials may be identified in this document in order to describe an experimental procedure or concept adequately. Such identification is not intended to imply recommendation or endorsement by the National Institute of Standards and Technology, nor is it intended to imply that the entities, materials, or equipment are necessarily the best available for the purpose.\\ 
% \vfill
% %%%%%%%%%%%%%%%%%%%%%%%%%%%%%%%%%%%%%%%%%%%%%%%%%%%%%%%%%%%%%%%%%%%%
% %   This secton automated - do not change
% %%%%%%%%%%%%%%%%%%%%%%%%%%%%%%%%%%%%%%%%%%%%%%%%%%%%%%%%%%%%%%%%%%%%
% \normalsize \textbf{National Institute of Standards and Technology Technical Note \pubnumber\\ 
% Natl. Inst. Stand. Technol. Tech. Note \pubnumber, \pageref{LastPage} pages (\monthyear)} \\
% \textbf{CODEN: NTNOEF}\\
% \vspace{12pt}
% \textbf{This publication is available free of charge from: \DOI}
% \vfill
% \end{flushright}
% \end{titlepage}
%%%%%%%%%%%%%%%%%%%%%%%%%%%%%%%%%%%%%%%%%%%%%%%%%%%%%%%%%%%%%%%%%%%%
%   Start front matter - page number starts with "i"
%%%%%%%%%%%%%%%%%%%%%%%%%%%%%%%%%%%%%%%%%%%%%%%%%%%%%%%%%%%%%%%%%%%%
% \section*{Foreword}
% \pagenumbering{roman}
% \normalsize Delete if not applicable\\
% \section*{Preface}
% \normalsize Delete if not applicable\\
% \section*{Abstract}
% \normalsize Required\\
% \section*{Key words}
% \normalsize Required, alphabetized, separated by semicolon, and end in a period.\\
% \pagebreak
%%%%%%%%%%%%%%%%%%%%%%%%%%%%%%%%%%%%%%%%%%%%%%%%%%%%%%%%%%%%%%%%%%%%
%   Table of Contents is required
% 	List of Tables & Figures required if more than 5 tables/figures
%%%%%%%%%%%%%%%%%%%%%%%%%%%%%%%%%%%%%%%%%%%%%%%%%%%%%%%%%%%%%%%%%%%%
\begin{center}
	\tableofcontents
% 	\listoftables
% 	\listoffigures
\end{center}
\pagebreak
% \section*{Glossary}
% Delete if not applicable\\
% \pagebreak
%%%%%%%%%%%%%%%%%%%%%%%%%%%%%%%%%%%%%%%%%%%%%%%%%%%%%%%%%%%%%%%%%%%%
%   Start body of text - page number starts with "1"
%%%%%%%%%%%%%%%%%%%%%%%%%%%%%%%%%%%%%%%%%%%%%%%%%%%%%%%%%%%%%%%%%%%%
\section{Introduction}
\label{sec:intro}
\pagenumbering{arabic}
\normalsize This project involves designing and implementing a new system, called Peer-to-Peer Database (P2PDB), that will serve as a decentralized platform for data publishing, sharing, and querying of data. P2PDB enables an unlimited number of independent participants to publish and access the contents of datasets stored across the participants.

\subsection{Roles in System}

\begin{itemize}
    \item \textbf{Publisher}: produce and ship its structure data to contractors for storage and queryable access, and receive a reward every time the data that they contributed participates in a query result.
    \item \textbf{Client}: Produce static or continuous SQL queries and consume the results.
    \item \textbf{Coordinator}: receive SQL queries, parse, plan, optimize  and coordinate their parallel execution across potentially many contractors.
    \item \textbf{Contractor}: store all data in the peer-to-peer network, and provide an interface to locally stored data for queries sent from coordinators.
    \item \textbf{Blockchain}: maintain the configuration and bookkeeping info of the global state of sytem.    
\end{itemize}
           
\subsection{Query Flow}


\begin{enumerate}
\item \textbf{Client} read data from a \textbf{blockchain} and help clients generate SQL queries. Then, send SQL queries to its preferred \textbf{coordinator}.
\item \textbf{Coordinators} are constantly following updates on the \textbf{blockchain}, in order to be aware of all \textbf{contractor} groups, \textbf{publisher} groups, and \textbf{schemas} that exist. The coordinator parses, optimizes and generates a query plan for the query. The query is then performed in parallel (using standard parallel query processing techniques) across all of the confused \textbf{contractors} which receive the reward from the \textbf{coordinator} in return for its effort. 
\item \textbf{Coordinator} aggregates results, receive the reward from \textbf{client}.
\end{enumerate}

\section{Early Thoughts}

The core (technical) components of P2PDB are contractor and blockchain. Because the system must be robust in the face of malicious nodes (\textbf{contractor}) that join the system and intentionally corrupt data or return incorrect results upon receiving a query. In other words, all contractors in the byzantine environment with large numbers of contractors frequently joining and leaving the network. 

\noindent \textbf{Question 1: Shall we build contractors as a peer-to-peer network?}

In AnyLog paper, the coordinator and contractor is a client/server architecture. 
There is no association between the contractors where each contractor is an independent, self-aware node equipped with a time series database system such as TimeScaleDB \cite{timescaledb} and InfluxDB \cite{influxdb} by guaranteeing the IoT data immutability. However, in addition to the client–server model, distributed computing applications often use the peer-to-peer (P2P) application architecture. It's better to build a P2P network for contractors who will be starting to become \textbf{aware of each other}. There's no harm in that because both client-server and master-slave are regarded as sub-categories of distributed P2P systems. For now, we don't see any requirement needs contractors communicate with each other directly. But, eventually it will become the underlying trends shaping P2PDB more robust, scalable, and flexible. For example, some publisher may contribute their data to multiple contractors to provide a solid SLA. The rewards from coordinator are shared among these contractors. If a contractor becomes unavailable, its shared resources remain available as long as other peers offer it. Contractors with modest resources can help to share the load. \\

\noindent \textbf{Question 2: Shall we separate blockchain as an individual subsystem? Is that a permissioned or permissionless blockchain?}

Yes. We don't want to integrate blockchain node into contractors to slow queries down. Nowadays blockchain systems like Ethereum \cite{ethereum} makes an intensive use of syscalls which interrupt the CPU a lot \cite{analyze_eth} and consumes lots of RAM. The blockchain in P2PDB serves as a global log of all configuration and meta information. Though the blockchain component, contractors and publishers can register their existence. In addition, any entitiy may register a schema on the blockchain. Blockchain in here is also used for registering payment channels. 

Depends on the purpose of P2PDB, we may fork one or two popular advanced blockchain systems which are the most advanced for coding and processing smart contracts. For example, Ethereum as a public blockchain in P2PDB to provide general-purpose data sharing solution. Libra and Hyperledger as private/permissioned blockchains that need special permissions to read, access, and write information on them such as medical data
sharing where all the decisions are taken by the leader or a board of decision makers.


\section{How to Start it from Scratch}

All research efforts such as query verification, secure data processing, decentralized dispute moderation (violations to SLA) and even query optimization and execution over a federated architecture depend on contractor's implementation. We better start from building contractors.

\begin{enumerate}
\item Build a peer-to-peer network for contractors to join and leave P2PDB. Bitcoin use a gossip protocol, which is a procedure or process of computer peer-to-peer communication that is based on the way epidemics spread.  The Storj V3 network, Ethereum, BitTorrent, Swarm, and IPFS utilizes Kademlia DHT protocol \cite{Maymounkov:2002:KPI:646334.687801} to provide a way for millions of computers to self-organize into a network, communicate with other computers on the network, and share resources (e.g. files, blobs, objects) between computers, all without a central registry or lookup run by a single person or company. Since open source communities have many progresses/efforts in recent years, we also adopt Kademlia DHT to build P2P network for contractors. Building large scale peer-to-peer systems has been complex and difficult in the last 15 years, and \textbf{libp2p} \cite{libp2p} is a way to fix that. It is a "network stack" -- a protocol suite -- that cleanly separates concerns, and enables sophisticated applications to only use the protocols they absolutely need, without giving up interoperability and upgradeability. We may use dht module \cite{libp2p-dht} only from libp2p.
Libra's networking protocol \cite{libra-doc} is inspired by the libp2p project. 

\item Integrate a time series database system into each contractor node.

\item Implement blockchain subsystems to store global states of P2PDB. Need more time to investigate smart contracts for this part. 

\item Publisher and Client are just bunch of RPC calls and Blockchain registrations.

\end{enumerate}

\section{Reseach Opportunities}

\begin{itemize}
\item  Violations to SLA are handled via a decentralized dispute mechanism. This dispute mechanism is implemented via smart contract that run in the Blockchain. This dispute smart contract has the capability of performing actions such as charging a fine to the offending party. Futhermore, mechanisms like ADSs and TEEs (SGX) are capable of enforcing the integrity guaranteees of an SLA.
\item Query verification, secure data processing and query optimization and execution over a federated architecture.
\item Implement entire P2PDB with the decentralized compensation mechanism is also a research opportunity for a great system paper.
\item A publisher/contractor might represent a group of publishers/contractors acting as a unit. Contractor groups can usually offer better SLAs and service guarantees by overriding the the same interfaces and providing its own custom federated query implementation. For example, there's definitely lots of SQL aggregate functions be called inside contractor groups. If someone deploys a contractor group in datacenter and aims to meet strict SLAs to attract more clients, they can do in-network processing and data aggregation through programmable switches. That's an opportunity to collabrate with some networking folks to optimize query over a federated database. In a similar way, coordinator also aggregate results.

\item After we start to build P2PDB and dig deeper, we are more likely to come up with much more research ideas. 

\end{itemize}

%%%%%%%%%%%%%%%%%%%%%%%%%%%%%%%%%%%%%%%%%%%%%%%%%%%%%%%%%%%%%%%%%%%%
% \begin{table}[H]
% 	\centering
% 	\caption{Title.}
% 	\small
% 	\begin{tabular}{cc}
% 		\hline
% 		ColumnA & ColumnB \\ \hline
% 		text & text{\scriptsize $^{\rm a}$} \\
% 		text & text \\
% 		text & text \\
% 		text & text \\
% 		\hline
% 	\end{tabular}
	
% 	{\footnotesize 	{\scriptsize $^{\rm a}$}Footnote}
% \end{table}

% \begin{figure}[h] 
% 	\centering 	\includegraphics[width=0.5\linewidth]{Chrysanthemum.jpg}
% 	\caption{This is the caption text.}
% 	\label{fig:Chrysanthemum}
% \end{figure}
%%%%%%%%%%%%%%%%%%%%%%%%%%%%%%%%%%%%%%%%%%%%%%%%%%%%%%%%%%%%%%%%%%%%

% \section*{Acknowledgments}
% \noindent Delete if not applicable\\
%%%%%%%%%%%%%%%%%%%%%%%%%%%%%%%%%%%%%%%%%%%%%%%%%%%%%%%%%%%%%%%%%%%%
%   Acknowledgments not required
%%%%%%%%%%%%%%%%%%%%%%%%%%%%%%%%%%%%%%%%%%%%%%%%%%%%%%%%%%%%%%%%%%%%

\section*{References}
\addcontentsline{toc}{section}{References}
\bibliographystyle{techpubs}
\bibliography{References}

%%%%%%%%%%%%%%%%%%%%%%%%%%%%%%%%%%%%%%%%%%%%%%%%%%%%%%%%%%%%%%%%%%%%
%   Please use the techpubs BibTeX style when compiling bibliography, or follow the instructions on tinyurl.com/techpubsnist to format your .bib / .bbl file appropriately.
%%%%%%%%%%%%%%%%%%%%%%%%%%%%%%%%%%%%%%%%%%%%%%%%%%%%%%%%%%%%%%%%%%%%

% \section*{Appendix A: Supplemental Materials}
% \addcontentsline{toc}{section}{Appendix A: Supplemental Materials}
% Brief description of supplemental files\\

% \section*{Appendix B: Change Log}
% \addcontentsline{toc}{section}{Appendix B: Change Log}
% If updating document with errata, detail changes made to document – delete if not applicable. \\

\end{document}
